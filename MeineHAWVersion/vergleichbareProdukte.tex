\chapter{Vergleichbare Produkte}
Da ein Blick auf die im Markt etablierten Produkte nur f�rderlich ist, um herauszufinden in wie weit 

\section{VirtualBox}
Die im April 2005 entstandene Virtualisierungssoftware von Oracle (erst InnoTek Systemberatung GmbH) eignet sich f�r Windows, Linux, Mac OSX, FreeBSD und Solaris als Wirtssystem. Als Gastsysteme k�nnen sowohl x64- als auch x86-Betriebssysteme eingesetzt werden. \newline
VirtualBox verwendet das eigene Format VDI (Virtual Disk Images)


Festplatten werden in Containerdateien, von VirtualBox auch als ?virtuelle Plattenabbilder? (englisch Virtual Disk Images, kurz VDI) bezeichnet, emuliert. Neben diesem eigenen Dateiformat kann VirtualBox auch mit Festplattendateien von VMware-Virtualisierungsprodukten (mit der Dateiendung ?.vmdk?), dem ?Virtual Hard Disk?-Format (mit der Dateiendung ?.vhd?) von Windows Virtual PC, HDD-Dateien von Parallels sowie mit Abbildern im QED- (QEMU enhanced disk) und QCOW-Format (QEMU Copy-On-Write) der Emulations- und Virtualisierungssoftware QEMU umgehen. Zudem k�nnen iSCSI-Objekte als virtuelle Festplatten genutzt werden, wobei der hierf�r ben�tigte iSCSI-Initiator bereits in VirtualBox enthalten ist. Mit dem zu VirtualBox geh�renden Kommandozeilen-Werkzeug VBoxManager kann man diese fremden Formate auch konvertieren.



\section{VMare}
\section{Opennebula}

