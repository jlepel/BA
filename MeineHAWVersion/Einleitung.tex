\chapter{Einleitung}

\begin{quote}
	``Es ist nicht zu wenig Zeit, die wir haben, sondern es ist zu viel Zeit, die wir nicht nutzen'' - Lucius Annaeus Seneca, \cite{Apelt200511}
\end{quote}
Seneca formulierte  49 n. Chr. das Gef"uhl welches jeder kennt. Die Zeit die er hat, nicht richtig zu nutzen.
Technische Neuerungen helfen uns unsere Zeit besser zu planen, mehr Zeit in andere Aktivit"aten zu stecken und unsere Priori"aten zu "uberdenken.
Diese Arbeit besch"aftigt sich mit dem Teil-Aspekt der Informatik, der virtualisierung von Servern im Entwicklungsumfeld.\newline
...

\section{Problemstellung}
%Die Einleitung muss Ihr Thema eingrenzen (und diese Eingrenzung rechtfertigen) und Ihr Erkenntnisinteresse präzisieren und begründen
Virtualisierung hat in vielen Bereichen den physichen Server abgel"ost. \newline
Denn der Finanzielle Aspekt ist f"ur Unternehmen nicht unerheblich. Im Idealfall heisst der Umstieg auf virtuelle Landschaften gleich weniger Server, was gleichbedeutend mit weniger Stellfl"ache ist. Somit auch mit weniger Racks und weniger Verkabelung.\newline
Aufw"andige Vorplanung von Serverzentren entf"allt, die Kostenplanung der unterschiedlichen Hardware wird minimiert und die Frage, was in ein paar Jahren mit der Harware passieren soll, wird obsolet.\newline
Gerade im Entwicklungsbereich ist es mei{\ss}t sinnvoller virtuelle Umgebungen zu realisieren, als reale Maschinen aufzubauen. Entwickler haben so die M"oglichkeit bei Bedarf sich Abz"uge der Produktionumgebung zu erstellen oder Fehlerszenarien nachzustellen.\newline
Meisst ist dazu die Involvierung des Betriebs-Teams oder des IT-Support notwendig, die nach Priorit"at ihrer Auftragslage, eine gewissen Vorlaufszeit ben"otigen um die gew"unschte Maschine aufzubauen. \newline
In dem Fall, dass die Firmengr"o{\ss}e es nicht erlaubt, eine eigene Support-Abteilung zu haben, muss die Zeit des jeweiligen Mitarbeiters herhalten, um das Wissen "uber die jeweilige Virtualisierungsl"osung aufzubauen, die gew"unschte Maschine zu erstellen und die Installationen der n"otigen Programme zu realisieren. Der R"uckschluss daraus ist, geringere Produktivit"at in den Kernt"atigkeiten des Mitarbeiters. \textbf{TODO: Gef"allt mir noch nicht so ganz.}\newline
Auch wenn die Softwarebranche eine Vielfalt an M"oglichkeiten bereitstellt, sind Diese entweder in ihrer Struktur zu "uberdimensioniert, um sie in der Anwendung schnell zu erlernen, oder komplex in ihrer Konfiguration im Bezug auf Automatisierungen und/oder Provisionierungen.

%Da die Softwarebranche vielf"altige M"oglichkeiten bereitstellt, um mit geringem Aufwand lokal eine virtuelle Umgebung zu erzeugen und diese ggf. zu Provisionieren. 
%Im idealfall heisst das f"ur Unternehmen: Weniger Server sind gleichbedeutend mit weniger Stellf"ache, mit weniger Verkabelung oder Racks. Somit ist die %Konsolidierung der ehemals gro{\ss}en Server-Zentren f"ur viele Unternehmen eine direkte Konsequenz. 
%Wodurch eine Kostenreduzierung der gesamten Infrastruktur entsteht.\newline
%Nicht nur der Finanzielle Aspekt spricht oft f"ur die Virtualisierung, sondern auch die leichte Automatisierung, die Erh"ohung der Verf"ugbarkeit und ....\newline
%Das Verschieben von kompletten Applikationen von einem physischen Ort zu einem anderen,....\newline
%Verbesserung der Verf"ugbarkeit und Business Continuity. Dazu geh"oren Live Migration, Storage Migration, Fehlertoleranz, Hochverf"ugbarkeit und Ressourcen-Management. Virtuelle Maschinen k"onnen damit leicht verschoben und vor ungew"unschten Auszeiten gesch"utzt werden.\newline
%In der physischen Welt war es bisher "ublich, jeder Applikation einen eigenen Server zuzuweisen. Damit war daf"ur Sorge getragen, dass die einzelnen Software-Programme sauber voneinander isoliert waren. Aber das f"uhrte auch zu einem Wust von Rechnern, von denen viele noch dazu nicht optimal ausgelastet waren. Und die Kosten f"ur diese Server-Landschaft liefen schnell aus dem Ruder. Nicht so bei Virtualisierung. Inzwischen sind auch die n"otigen Funktionen und Tools vorhanden, um VMs und die in ihnen verpackten Anwendungen sauber voneinander zu trennen. CPU, Memory und Storage k"onnen exakt ausgelastet werden, die Kosten in einem solchen Modell sinken.\newline


\section{Zielsetzung}
Ziel der vorliegenden Arbeit ist es, ein Softwareprodukt zu erarbeiten, welches aktuelle Virtuallisierungs-, sowie Provisionierungsl"osungen verwendet, um mit deren Hilfe den Aufbau von tempor"aren (ad hoc) Umgebungen im virtuellen Umfeld zu vereinfachen.\newline
Eine Auswahl von leicht erlernbaren und unkomplizierten Softwarekomponenten, f"ordern ein weiteres Ziel. Den Administrationsaufwand gering wie m"oglich zu halten und auch Linux/Unix unerfahrene Administratoren, sowie Entwickler anzusprechen.\newline
Es ist angestrebt, bei Ende dieser Arbeit eine zentralisierte und leicht lokal zu implementierende Anwenundung zur Verf"ugung zu stellen.
Sie soll es dem Benutzer erm"oglichen, sich selbstst"andig und mit geringem Zeitwand eine virtuelle Maschine mit gew"unschter Software zu erstellen, ohne gro{\ss}e Einarbeitung in Benutzung und Konfiguration. \newline
Somit wird die administrative Instanz des z.B Unternehmen entlasten und bef"ahigt den Benutzer sich auf seine Kernt"atigkeiten zu konzentrieren.

\section{Motivation}
Die Motivation dieser Arbeit besteht darin, eine Software zu entwicklen, die durch vereinfachte Handhabung und minimaler Einarbeitungszeit, es dem Benutzer erm"oglich eine ad-hoc Umgebung zu erstellen, ohne b"uroktratischen Aufwand und ohne Grundwissen "uber die darunterliegende Anwendungsstruktur.
Der normalerweise gro{\ss}e zeitliche Aufwand soll m"oglichst minimiert werden und es Anwendern in Unternehmen und Projekten erleichtert wird, sich auf die vorhandenen Usecase zu fokussiern und keine Zeit in Aufbau, Installation und Problembehebung investieren zu m"ussen.

\section{Themenabgrenzung}
Diese Arbeit greift bekannte und etablierte Softwareprodukte auf und nutzt Diese in einem zusammenh"angenden Kontext. Dabei werden die verwendeten Softwareprodukte nicht modifiziert, sondern f"ur eine vereinfachte Benutzung durch eigene Implementierungen kombiniert und mit einem Benutzerinterface versehen, welches die Abl"aufte visuallisiert und dem Benutzer die Handhabung vereinfacht.
Die vorzunehmenden Implementierungen greifen nicht in den Ablauf der jeweiligen Software ein, sondern vereinfacht das Zusammenspiel der einzelnen Anwendungen.

\section{Struktur der Arbeit}

