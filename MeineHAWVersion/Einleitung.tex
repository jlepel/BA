\chapter{Einleitung}

\begin{quote}
	``Es ist nicht zu wenig Zeit, die wir haben, sondern es ist zu viel Zeit, die wir nicht nutzen'' - Lucius Annaeus Seneca, \cite{Apelt200511}
\end{quote}

Seneca formulierte  49 n. Chr. das Gef"uhl welches jeder kennt. Die Zeit die er hat, nicht richtig zu nutzen.
Technische Neuerungen helfen uns unsere Zeit besser zu planen, mehr Zeit in andere Aktivitäten zu stecken und unsere Priori"aten zu überdenken.
Diese Arbeit beschäftigt sich mit dem Teil-Aspekt der Informatik, der virtualisierung von Servern im Entwicklungsumfeld.

\section{Problemstellung}
Server-Virtualisierung ist in der heutigen Zeit keineswegs mehr eine Seltenheit. Sie ist eher Standard in den mei{\ss}ten Unternehmen.\newline
Virtualisierung hat in vielen Bereichen den physichen Server abgelöst, denn es müssen keine aufwändigen Planung im Vorfeld getroffen werden, der Einkauf muss nicht mehr involviert werden, das Budget braucht nicht kalkuliert werden und die Frage, was in ein paar Jahren mit der Hardware passiert, wird obsolet.

Im idealfall heisst das für Unternehmen: Weniger Server sind gleichbedeutend mit weniger Stellf"ache, mit weniger Verkabelung oder Racks. Somit ist die Konsolidierung der ehemals gro{\ss}en Server-Zentren f"ur viele Unternehmen eine direkte Konsequenz. 
Wodurch eine Kostenreduzierung der gesamten Infrastruktur entsteht.\newline
Nicht nur der Finanzielle Aspekt spricht oft f"ur die Virtualisierung, sondern auch die leichte Automatisierung, die Erh"ohung der Verf"ugbarkeit und ....\newline
Das Verschieben von kompletten Applikationen von einem physischen Ort zu einem anderen,....\newline
Verbesserung der Verf"ugbarkeit und Business Continuity. Dazu geh"oren Live Migration, Storage Migration, Fehlertoleranz, Hochverf"ugbarkeit und Ressourcen-Management. Virtuelle Maschinen k"onnen damit leicht verschoben und vor ungew"unschten Auszeiten gesch"utzt werden.\newline
In der physischen Welt war es bisher "ublich, jeder Applikation einen eigenen Server zuzuweisen. Damit war daf"ur Sorge getragen, dass die einzelnen Software-Programme sauber voneinander isoliert waren. Aber das f"uhrte auch zu einem Wust von Rechnern, von denen viele noch dazu nicht optimal ausgelastet waren. Und die Kosten f"ur diese Server-Landschaft liefen schnell aus dem Ruder. Nicht so bei Virtualisierung. Inzwischen sind auch die n"otigen Funktionen und Tools vorhanden, um VMs und die in ihnen verpackten Anwendungen sauber voneinander zu trennen. CPU, Memory und Storage k"onnen exakt ausgelastet werden, die Kosten in einem solchen Modell sinken.
...

\section{Zielsetzung}
Ziel der vorliegenden Arbeit ist es, eine Softwareprodukt zu erarbeiten, die es erm"oglicht zeiteffizient tempor"are virtuelle Umgebungen inklusive gew"unschten zugeh"origen Programmen aufzubauen und deren Administration leicht verst"andlich und unkompliziert aufgebaut ist.
Dies soll durch Verwendung von aktuellen Softwareprodukten geschehen, die für diesen Bereicht im Markt etabliert sind.\newline
....

\section{Struktur der Arbeit}

\section{Themenabgrenzung}
Diese Arbeit greift Bekannte und etablierte Softwareprodukte im auf und nutzt diese in einem zusammenh"angenden Kontext. Dabei wird keine der verwendeten Software modifiziert, sondern f"ur eine vereinfachte Benutzung kombiniert und mit einem Benutzerinterface versehen, welches die Abl"aufte visuallisiert und dem Benutzer die Handhabung vereinfacht.


-----Im folgenden wird darauf eingegangen, ob und wie es m"oglich ist, durch Verwendung von aktuellen Softwareprodukten, sowie deren automatisiertem zusammenspiel, eine eigenständige Software zu entwicklen, die zeitsparend agiert----










