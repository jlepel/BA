\section{Vagrant}
Bei Vagrant handelt es sich um ein Softwareprojekt, welches 2010 von Mitchell Hashimoto und John Bender 2010 ins Leben gerufen wurde.
Der prim"are Gedanke hinter diesem Projekt ist es, gerade Entwicklern und Teams eine schnelle und unkomplizierte M"oglichkeit zu bieten, virtuelle Maschinen und Landschaften zu erstellen.\newline
Vagrant ist also ein m"achtiges Werkzeug zum virtuallisieren, der sonst oft lokalen Entwicklungsumgebungen. 
Gerade Teamarbeit wird dadurch vereinfacht, denn die gew"unschten Maschinen k"onnen mit den gleichen Konfigurationen, Komponenten, Infrastrukturen und Bibliotheken erstellt werden.\newline 
Um die gew"unsche Maschine zu visuallisieren, greift Vagrant auf VirtualBox zur"uck.
VirtualBox ist Oracles Freeware Pandant zur kommerziellen VMware Produktlinie. Wenn gew"unscht, kann auf VMware Fusion oder Amazon Web Services zur"uckgegriffen werden.\newline 
Die Konfiguration einer Maschine geschieht "uber das "Vagrantfile"; in dem Parameter wie IP Adresse konfiguriert oder Provisionierer hinzuschaltet.
Da das Vagrantfile in einer Ruby Domain Specific Language geschrieben wird, bedeutet das f"ur den Anwender, dass er es einfach mit anderen Kollegen "uber Versionskontrollen (z.B. Git oder Subversion) teilen kann.
Bei den Provisionierern wird dem Benutzer die Freiheit gegeben, auf Bekannte wie Chef, Puppet oder Ansible zur"uckzugreifen.

\section{Konfiguration}
Vagrantfile beinhalten die Konfiguration jeder Vagrantmaschine. Sogar jeder Vagrant-"Landschaft". 
Das in Ruby Syntax geschriebene Konfigurationsfile, wird automatisch "uber den Befehl "vagrant init" im gewünschten Ordner generiert oder manuell über jeden Editor erstellt werden. Auf Linux-Systemen ist darauf zu achten, dass der gewünschte Ordner "uber entsprechende Berechtigungen verf"ugt.
Manuelle Erweiterung des Vagrantfiles wird durch die Rubysyntax zus"atzlich vereinfacht.









