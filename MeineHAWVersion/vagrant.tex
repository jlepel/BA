\section{Vagrant}
Bei Vagrant handelt es sich um ein Softwareprojekt, welches 2010 von Mitchell Hashimoto und John Bender 2010 ins Leben gerufen wurde.
Der prim�re Gedanke hinter diesem Projekt ist es, gerade Entwicklern und Teams eine schnelle und unkomplizierte M�glichkeit zu bieten, virtuelle Maschinen und Landschaften zu erstellen.\newline
%Vagrant ist also ein m�chtiges Werkzeug zum virtualisieren, der sonst oft lokalen Entwicklungsumgebungen. 
%Gerade Teamarbeit wird dadurch vereinfacht, denn die gew�nschten Maschinen k�nnen mit den gleichen Konfigurationen, Komponenten, Infrastrukturen und Bibliotheken erstellt werden.\newline 
Standardm��ig greift Vagrant auf VirtualBox zur�ck, um die Virtualisierung vorzunehmen. 
VirtualBox ist Oracles Freeware Pendant zur kommerziellen VMware Workstation/Fusion. Die Installation von Plugins erm�glicht es, statt auf VirtualBox auf VMware Workstation/Fusion oder Amazon Web Services zur�ckgegriffen werden.\newline 
Die Konfiguration einer Maschine geschieht �ber das 'Vagrantfile', in dem Parameter wie IP Adresse konfiguriert oder Provisionierer hinzuschaltet werden k�nnen. Bei den Provisionierern wird dem Benutzer die Freiheit gegeben, auf Bekannte wie Chef, Puppet oder Ansible zur�ckzugreifen.\newline
Da das Vagrantfile in einer Ruby Domain Specific Language geschrieben wird, bedeutet das f�r den Anwender, dass er es einfach mit anderen Kollegen �ber Versionskontrollen (z.B. Git oder Subversion) teilen kann.\newline
Abgesehen von dem Austausch der Konfigurationen, wird die Teamarbeit durch eine 'sharing' Option unterst�tzt, die mit Version 1.5 implementiert wurde.
Das Teilen eine Maschine erm�glicht es Teams an gemeinsamen und entfernten Standorten auf die gleiche Maschine zuzugreifen. 

\subsection{Konfiguration}
Vagrantfile beinhalten die Konfiguration jeder Vagrantmaschine. Sogar jeder Vagrant-"Landschaft". 
Das in Ruby Syntax geschriebene Konfigurationsfile, wird automatisch �ber den Befehl 'vagrant init' im gew�nschten Ordner generiert oder manuell �ber jeden Editor erstellt werden. Auf Linux-Systemen ist darauf zu achten, dass der gew�nschte Ordner �ber entsprechende Berechtigungen verf�gt.
Manuelle Erweiterung des Vagrantfiles wird durch die Rubysyntax zus�tzlich vereinfacht.


\subsection{Vergleich zu Docker}
%Im Zusammenhang mit ad hoc Umgebungen, liegt der Fokus in Diskussion oft auf Docker und Vagrant.
%Der Grundgedanke bei beiden Applikationen ist der Gleiche, aber man muss sich bewusst sein, was das Ziel der Anwendung sein soll.
%Beide Applikationen haben wie alles ihre Vor- und Nachteile, aber in einem sind beide Virtualisierungs-Tools gleich. Die zentrale Steuerung zum Aufbau einer %Maschine, geschieht �ber ein einziges Konfig-File.\newline
Docker ist ein Linux-only VE (Virtual Environment)-Tool und arbeitet im Gegensatz zu Vagrant mit Operating-System-Level Virtualisierung, auch Linux Containern (LxC) genannt. W�hrend Vagrant Hypervisor-basierten virtuellen Maschinen aufbaut.
F�r die sogenannten Container nutzt Docker spezielle Kernelfunktionalit�ten, um einzelne Prozesse in Containern voneinander zu isolieren.\newline 
Dadurch wird f�r den Benutzer der Eindruck erweckt, dass f�r Prozesse die mit Containern gestartet werden, diese auf ihrem eigenen System laufen w�rden.
Docker wird in drei Teile unterteilt. Die Arbeitsweise ist nicht wie bei einer VM (virtuelle machine) bei der ein virtueller Computer mit einem bestimmten Betriebssystem, Hardware-Emulation sowie Prozessor simuliert wird. Ein VE ist quasi eine leichtgewichtige virtuelle Maschine. In einem Container ausgef�hrte Prozesse greifen gemeinschaftlich auf den Kernel des Hostsystems zu. Durch die Kernelnamensr�ume(cnames) werden die ausgef�hrten Prozesse voneinander isoliert. Allerdings ist gerade die Isolation der Prozesse, durch den gemeinsam genutzten Kernel etwas geringer, als bei der Benutzung durch einen Hypervisor.\newline
Durch die zugrunde liegende Architektur von Vagrant, wird Vagrant gerne f�r immer gleich aufbauende Entwicklungsumgebungen benutzt.
Die Vielzahl an unterst�tzten Betriebssystemen macht es f�r viele Benutzer attraktiver. Allerdings kooperieren Vagrant und Docker auch hervorragend zusammen.
%Vagrant verspricht durch die leichtere Handhabung und Konfiguration eher das, was f�r das vorliegende Projekt entscheidend ist.



%ZITAT: Docker wird in diversen Varianten als L�sung f�r das Setup und die Kapselung lokaler Entwicklungsumgebungen, als tempor�r verf�gbarer Service imRahmen von Integrationstests und als Laufzeitumgebung auf Produktionsservern eingesetzt. Gerade der tempor�re Charakter von Docker-Containern wird f�r einen anderen Anwendungsfall interessant: als dynamisch erzeugte Buildsysteme mit einer projektspezifischen Buildumgebung.
%Quelle: http://www.scriptrock.com/articles/docker-vs-vagrant 
%Quelle: https://entwickler.de/online/docker-basics-system-level-virtualisierung-125514.html






