\section{Ruby}
Die Programmiersprache Ruby wurde in etwa Mitte der 90er Jahre entwickelt und ist eine objektorientierte, dynamisch typisierte Sprache.
Sie zeichnet sich unter anderem durch ihre einfache und leicht erlernbare Syntax aus und ihrer einfachen Wartung des Quellcodes.
Bekannt geworden ist sie durch besonders durch das Framwork Ruby on Rails.\newline
Ruby Unterst"utzt diverse Programmierparadigmen, wie unter anderem prozedurale und funktionale Programmierung sowie Nebenl"aufigkeit und wird direkt von einem Interpreter verarbeitet. Ausserdem bietet die Sprache eine automatische Garbage Collection, Regul"are Audr"ucke, Exceptions, Bl"ocke als Parameter f"ur Iteratoren und Methoden, Erweiterung von Klassen zur Laufzeit und Threads.\newline 
Da Ruby nicht Typisiert ist, wird alles als ein Objekt angesehen. Auf primitive Datentypen wird g"anzlich verzichtet.
%Quelle: http://b-simple.de/download/ruby.pdf
%Quelle: http://openbook.rheinwerk-verlag.de/it_handbuch/09_004.htm

\section{Sinatra}