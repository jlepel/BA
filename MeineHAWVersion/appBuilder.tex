\setcounter{secnumdepth}{3}
\chapter{Die Software}

\section{Idee und Anforderung}



\section{Ubersicht der Komponenten}
\subsection{Webkomponente}
\subsection{Vagrant}

Durch die Verwendung von Vagrant wird es m"oglich durch wenig Aufwand eine schnell verf"ugbare Umgebung aufzubauen.
Diese Arbeit bezieht sich in der Entwicklungsphase ausschliesslich auf eine Ubuntu 32Bit Version, die Online von Vagrant bereitgestellt wird.
So wird eine stabile Testbasis erzeugt, die vom Hersteller geprüft wurde.
Da Vagrant auf den g"angigen Systemen OS X, Windows und diversen Linux Distrubutionen installiert werden kann, es hervorragend mit diversen Provisionierern zusammenarbeitet und VM-Tools wie VirtualBox, VMWare und AWS unterst"utzt, wird es zu einem guten Allrounder. 
Gerade die Provisionierung von Software ,auf die zu startende virtuelle Maschine, macht Vagrant f"ur das vorliegende Projekt attraktiv.
Zwar dauert der Aufbau im Gegensatz zu Docker nicht Sekunden sondern Minuten, aber die Provisionierung ist mit einer der entscheidene Punkte f"ur die angestrebte Software.
Die leicht zu handhabende Konfiguration und die eing"angigen Begrifflichkeiten sind weiterer Punkte, die positiv f"ur die Benutzung in diesem Projekt sprechen.


\subsection{Ansible}
 Ansible.......


\section{Struktur und Zusammenspiel}
\section{Konfiguration}
\section{Datenbank}
\section{Virtualisierung}
\section{Provisionierung}
\section{Kommunikation der einzelnen Komponenten}
\section{Export Funktionen}

\subsection{Clonen einer Maschine}
Vagrant enth"alt nativ einen Befehlssatz um aus einer aktuell laufenden Maschine eine wiederverwendbare Vagrant-Box zu erstellen.
Um diese Box sp"ater wieder benutzen zu k"onnen, muss diese auf ein Laufwerk kopiert werden, welches von dem gew"unschten Mitarbeiter zugegriffen werden kann. Durch das kopieren auf dessen lokalen Datentr"ager und dem erstellen eines angepassten Import-Vagrantfiles, ist es m"oglich Maschinen an Mitarbeiter weiterzugeben.

http://www.dev-metal.com/copy-duplicate-vagrant-box/

\subsection{Export zu Git}
Die Vorraussetzung hier f"ur ist ein GitHub Konto, welches mit der Anwendung verkn"upft ist. 
Durch die Benutzung von GitHub kann das zuvor erstellte Vagrantfile automatisiert als Git-Repository hochgeladen werden.
Dies ermöglicht es Konfigurationen mit Mitarbeitern auszutauschen, zu archivieren oder zu versionieren.


\section{Sharing einer Maschine}
