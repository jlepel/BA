\chapter{Grundlagen}
Dieses Kapitel gibt einen Einblick in die verwendeten Programme, die f�r das Verst�ndnis der vorliegenden Arbeit von Vorteil sein k�nnen.
Es wird zun�chst auf die pr�gnanten Eigenschaften der Produkte eingegangen und wenn m�glich ein Vergleich mit Konkurrenzprodukten herangezogen.
Zudem wird teilweise die Handhabung und die Konfiguration kurz erl�utert.

\section{Ruby}
Die Programmiersprache Ruby wurde in etwa Mitte der 90er Jahre entwickelt und ist eine objektorientierte, dynamisch typisierte Sprache.
Sie zeichnet sich unter anderem durch ihre einfache und leicht erlernbare Syntax aus und ihrer einfachen Wartung des Quellcodes.
Bekannt geworden ist sie durch besonders durch das Framwork Ruby on Rails.\newline
Ruby Unterst�tzt diverse Programmierparadigmen, wie unter anderem prozedurale und funktionale Programmierung sowie Nebenl�ufigkeit und wird direkt von einem Interpreter verarbeitet. Au�erdem bietet die Sprache eine automatische Garbage Collection, Regul�re Ausdr�cke, Exceptions, Bl�cke als Parameter f�r Iteratoren und Methoden, Erweiterung von Klassen zur Laufzeit und Threads.\newline 
Da Ruby nicht Typisiert ist, wird alles als ein Objekt angesehen. Auf primitive Datentypen wird g�nzlich verzichtet.
%Quelle: http://b-simple.de/download/ruby.pdf
%Quelle: http://openbook.rheinwerk-verlag.de/it_handbuch/09_004.htm

\section{Sinatra} 